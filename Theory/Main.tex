\documentclass[12pt, letterpaper]{report}
\usepackage{amsmath}
% \usepackage{bm}
\title{Algorithm Theory}
\author{Jiayi Wu}
\begin{document}
    \maketitle
    \tableofcontents

    \chapter{Graph Matrix}
    Let G(V, E) be a graph, where $|V| = n$ and $|E| = m$. Assume G is unweighted and undirected, with no self loops or multiple edges.
    \section{Laplacian}
    \subsection{Adjacency Matrix}
    For graph G, let the adjacency matrix $A = A_{G}$ be defined as
    \begin{gather*}
        A_{G} = 
        \begin{cases}
            1 & (i, j) \in E\\
            0 & (i, j) \notin E
        \end{cases}
    \end{gather*}


    \subsection{Laplacian}
    The Laplacian of the graph, $L = L_{G}$ is defined as
    \begin{gather*}
        L_{G} = 
        \begin{cases}
            -1 & (i, j) \in E\\
            d_{i} & i = j\\
            0 & otherwise
        \end{cases}
        \\
        L_{G} = D_{G} - A_{G}
    \end{gather*}
    Let some vector \textbf{v} be thought of as a function that maps to the index to some value. ie.
    \begin{gather*}
        v_{i} = X(i)
    \end{gather*}
    $L_{G}\mathbf{v}$ can then be though as:
    \begin{gather*}
        [L_{G}\mathbf{v}]_i = d_i [X(i) - (\overline{X(j)}\mid (i, j) \in E)]
    \end{gather*}
    Let $\lambda_1 \leq \lambda_2 \leq \dots \leq \lambda_{n}$ be the eigenvalues of $L_G$ with corresponding eigenvectors $\mathbf{v_1}, \dots \mathbf{v_n}$. Then
    \begin{itemize}
        \item $\mathbf{v_{1}} = \mathbf{1}$, $\lambda_{1} = 0$
        \item All $\lambda_{i} \geq 0$
        \item Much information can be obtained by the first few non-trival \mbox{eigenvectors}.
    \end{itemize}


    \subsection{Properties of the Laplacian}
    \subsubsection{Edge Union} If G and H share the same vertex set have \textbf{disjoint} edge sets,
    \begin{gather*}
        L_{G \cup H} = L_G + L_H
    \end{gather*}
    \subsubsection{Isolated Vertex} If $d_i = 0$, then $[L_G]_{i,j} = 0$ and $[L_G]_{j,i} = 0$\\
    \subsubsection{Disjoint Union} If G and H have disjoint vertex sets, then
    \begin{gather*}
        L_{G \sqcup H} = L_G \oplus L_H = 
        \begin{pmatrix}
            L_G & 0\\
            0 & L_H
        \end{pmatrix}
    \end{gather*}
    If $L_G$ had eigenvectors $\mathbf{v_i}$ with eigenvalues $\lambda_i$ and $L_H$ had eigenvectors $\mathbf{u_i}$ with eigenvalues $\mu_i$, then $L_{G \sqcup H}$ has eigenvectors
    \begin{gather*}
        \mathbf{v_i}\oplus \mathbf{0}, \mathbf{0}\oplus \mathbf{u_i}
    \end{gather*}
    with the eigenvalues $\lambda_i$ and $\mu_i$ each.
    \subsubsection{Laplacian of Edge}
    The Laplacian of an edge $L_e$ has the form $\begin{pmatrix}
        1 & -1\\
        -1 & 1
    \end{pmatrix} = \begin{pmatrix}
        1\\-1
    \end{pmatrix}\begin{pmatrix}
        -1 & 1
    \end{pmatrix}$
    . The graph can be written as $\sum L_e$. Hence for vector \textbf{x} = $\begin{pmatrix}
        x_{1} \\
        x_{2}
    \end{pmatrix}$, $\mathbf{x^T} L_e \mathbf{x} = (x_1 - x_2)^2 \geq 0$. Hence the Laplacian is \textbf{positive semidefinite}. 
    This implies that all eigenvalues $\lambda_i \geq 0$ and that the matrix can be diagonalized as $L_e = Q^T \Lambda Q$.\\
    \begin{gather*}
        \mathbf{x^T} L \mathbf{x}=\sum_{(i,j)\in E}(x_i-x_j)^2
    \end{gather*}
    In addition, because L is positive semidefinite, all eigenvalues of L are non-negative, $L_e = A^T A$ (for some A).\\
    A need not be square or unique.

    \subsubsection{Factoring the Laplacian}
    The \textbf{Incidence Matrix} $\nabla_G = \nabla_{e, v} = \begin{cases}
        1 & e = (v, w), v < w\\
        -1 & e=(v,w), v>w\\
        0 & otherwise
    \end{cases}$\\
    Then $L_G = \nabla^T \nabla$.
    \subsubsection{Dimension of Nullspace}
    If G is connected, then the nullspace is $span\{\mathbf{1}\}$.\\
    \indent Thus, for connected graphs, $\lambda_2 \geq 0$\\
    \indent The dimensions of the nullspace is equal to the number of connected components of G.\\
    \subsubsection{Spectra of Common Graphs}
    \textbf{Complete Graph} has eigenvalue 0 with multiplicity 1 and eigenvalue n with multiplicity $(n - 1)$\\
    eigenvectors of \textbf{1} and $\{\mathbf{x} \mid \mathbf{x} \cdot \mathbf{1} = 0\}$\\
    \textbf{Ring Graph} with n vertices has eigenvectors
    \begin{gather*}
        x_k (u) = \sin{\frac{2 \pi k u}{n}}\\
        y_k (u) = \cos{\frac{2 \pi k u}{n}}
    \end{gather*}
    where the eigenvalue $x_k$ and $y_k$ is $2-2\cos{\frac{2 \pi k}{n}}$\\
    \textbf{Path Graph} with n vertices has the eigenvectors \begin{gather*}
        v_k (u) = \sin (\frac{\pi k u}{n} + \frac{\pi}{2n})
    \end{gather*}
    That has the same Eigenvalues as $R_{2n}$

    \subsubsection{Graph Products}
    \textbf{Definition} Let G=(V,E) and H=(W,F). The Graph $G \times H$ has vertex set $V \times W$ and edge
    set 
    \begin{gather*}
        ((v_1, w), (v_2, w)), \forall (v_1, v_2) \in E, \and w \in W\\
        ((v, w_1), (v, w_2)), \forall v \in V, \and (w_1, w_2) \in F
    \end{gather*}
    \indent The product of 2 paths graph is the grid graph.\\
    \textbf{Eigenvales} If $L_G$ has the eigenvalues $\lambda_i$ with the eigenvectors 
    $v_i$ and $L_H$ has the eigenvalues $\mu_i$ with the eigenvectors $w_i$, then the 
    eigenvectors of $L_{G \times H}$ are listed
    \begin{gather*}
        z_{ij}(v, w) = x_i(v) y_j(w)
    \end{gather*}
    with eigenvalues $\lambda_i + \mu_u$.

    \subsection{Why it is Called the Laplacian}
    If one is to discrete the derivative on graphs, as in treat the vertex as a function, then the 
    "Laplacian" would be $-L_G$

    \subsubsection{Laplacian Eigenvalues}
    \textbf{Sum of Eigenvalues} Given an n-vertex graph, where $d_{max} = \max{d_i}$, 
    \begin{gather*}
        \sum{\lambda_i}=\sum{d_i} \leq d_{max} n
    \end{gather*}
    This is obtained from the trace. If one removes $\lambda_1=0$, then
    \begin{gather*}
        \lambda_2 \leq \frac{\sum{d_i}}{n-1}\\
        \lambda_n \geq \frac{\sum{d_i}}{n-1}
    \end{gather*}
    
    \subsubsection{Bounds of $\lambda_2$ and $\lambda_{max}$}
    \textbf{Courant-Fischer Formula} For any $n\times n$ symmetric matrix A,
    \begin{gather*}
        \lambda_{1} = \min_{||x||=1} \mathbf{x^T}A\mathbf{x} = \min_{||x||\neq 0}\frac{\mathbf{x^T}A\mathbf{x}}{\mathbf{x^Tx}}\\
        \lambda_{max} = \max_{||x||=1}{\mathbf{x^T}A\mathbf{x}} = \max_{x\neq \vec{0}}\frac{\mathbf{x^T}A\mathbf{x}}{\mathbf{x^Tx}}
    \end{gather*}
    Let $S_k$ denote the span of $v_1 \cdots v_k$, and let $S_k^\perp$ denote the orthogonal complement of $S_k$
    \begin{gather*}
        \lambda_k = \min_{|x|=1,x\in S_k^\perp}x^TAx=\min_{x\neq 0,x\in S_k^\perp}\frac{x^TAx}{x^Tx}
    \end{gather*}
    \\
    \textbf{Rayleigh Quotient}
    Let G=(V,E) be a graph with Laplacian $L_{G}$, then
    \begin{alignat*}{2}
        &\lambda_1 = 0 & \mathbf{v_1} = \mathbf{1}\\
        &\lambda_2 = \min_{x\perp v_1,x\neq 0}\frac{x^T L_G x}{x^Tx} &=\min_{\sum x=0,x\neq 0} \frac{\sum_{(i,j)\in E}(x_i-x_j)^2}{\sum_{i \in V}x_i^2}\\
        &\lambda_{max} = \max_{x\neq 0}\frac{x^T L_G x}{x^Tx} &=\max_{x\neq 0} \frac{\sum_{(i,j)\in E}(x_i-x_j)^2}{\sum_{i \in V}x_i^2}
    \end{alignat*}
    One can use this to bound the eigenvalues of graphs easily.
    \section{Cut Graphs}
    \subsection{Cut Ratio}
    Let $e(S)$ of a cut $S-\overline{S}$ denote the number of edges between $S$ and $\overline{S}$\\
    \textbf{Cut Ratio} $\phi$ of a cut $S-\overline{S}$ is $\phi = \frac{e(S)}{\min(|S, |\overline{S}|)}$\\
    The \textbf{cut of minimum ratio} is the cut that minimizes $\phi(S)$.\\The \textbf{isoperimetric number} of a graph is the value of minimum cut.
    \subsubsection{Integer Function of the Cut Ratio}
    Associate every cut $S-\overline{S}$ with a vector $x\in \{-1, 1\}^n$, such that
    \begin{gather*}
        x_i = \begin{cases}
            1 & i \in S\\
            -1 & i \in \overline{S}
        \end{cases}
    \end{gather*}
    Thus,
    \begin{gather*}
        e(S) = \frac{1}{4} \sum_{(i,j)\in E}(x_i-x_j)^2
    \end{gather*}
    \begin{gather*}
        |S||\overline{S}| = \sum_{i,j \in V}[i \in S, j \in \overline{S}] = \frac{1}{4}\sum_{i<j}(x_i-x_j)^2\\
        \frac{n}{2}\min(|S|,|\overline{S}|)\leq |S||\overline{S}| \leq n \min{(|S|, |\overline{S}|)}
    \end{gather*}
    Hence:
    \begin{gather*}
        \frac{1}{n}\phi(G) \leq \min_{x\in \{1,-1\}^n}\frac{\sum_{(i,j)\in E}(x_i-x_j)^2}{\sum_{i<j}(x_i-x_j)^2} \leq \frac{2}{n}\phi(G)
    \end{gather*}
    This allows approximation of $\phi(G)$ to a factor of 2, but NP-hard. Can change the range of x to be $[-1, 1]^n$, this is solvable.\\
    \textbf{Relaxation}\\
    \indent To find the min of function $f(x)$ on C (NP hard), relax the constraints $x\in C'\supseteq C$, let $f(p)$ and $f(q)$ be the min with 
    constraints C and C'. $f(p)\geq f(q)$. Round q to $q'\in C$ such that $f(q') \leq \gamma f(q)$. Thus $f(q')\leq\gamma f(q)\leq\gamma f(p)$\\\\
    \textbf{Bounding Min Cut}
    Relax $x\in\{-1, 1\}$ to $x\in R^n, \sum{x_i}=0$
    \begin{gather*}
        \sum_{i<j}(x_i-x_j)^2=n\sum{x_i^2}
    \end{gather*}
    \begin{gather*}
        \min_{x\in R^n}\frac{\sum_{(i,j)\in E}(x_i-x_j)^2}{\sum_{i<j}(x_i-x_j)^2} = \min_{x\in R^n, x\perp 1}\frac{\sum_{(i,j)\in E}(x_i-x_j)^2}{n\sum x_i^2}=\frac{\lambda_2}{n}
    \end{gather*}
    By definition of Courant-Fischer formula of the second eigenvalue. So,
    \begin{gather*}
        \phi(G) \geq \frac{\lambda_2}{2}
    \end{gather*}
    \subsection{Cheeger's Inequality}
    Upper bound on the minimum cut. These bounds are tight.
    \begin{gather*}
        \frac{\phi(G)^2}{2d_{max}}\leq \lambda_2 \leq 2\phi(G)
    \end{gather*}
    \subsubsection{Proof}
    For any $x\perp \mathbf{1}, x_1\leq x_2 \cdots x_n$, $\exists i$
    \begin{gather*}
        \frac{x^T Lx}{x^Tx} \geq \frac{\phi(\{1\cdots i\})}{2d_{max}}
    \end{gather*}
    Let
    \begin{gather*}
        m = \frac{n+1}{2}\\
        y_i=x_i-x_m
    \end{gather*}
    Then
    \begin{gather*}
        \frac{x^TLx}{x^Tx} \geq \frac{y^TLy}{y^Ty}
    \end{gather*}
    Next, if $(i,j)\in E, i<m<j$, then replace (i, j) with (i, m) and (m, j) in E'.
    \begin{gather*}
        \frac{\sum_{(i,j)\in E}(y_i-y_j)^2}{\sum_{i\in V}y_i^2} \geq \frac{\sum_{(i,j)\in E'}(y_i-y_j)^2}{\sum_{i\in V}y_i^2}
    \end{gather*}
    Split the latter on m,
    \begin{gather*}
        \frac{\sum_{(i,j)\in E_{-}'}(y_i-y_j)^2+\sum_{(i,j)\in E_{+}'}(y_i-y_j)^2}{\sum_{i\in [1,m]}y_i^2+\sum_{i\in [m,n]}y_i^2}
    \end{gather*}
    which has lower bound
    \begin{gather*}
        \frac{\sum_{(i,j)\in E_{-}'}(y_i-y_j)^2}{\sum_{i\in [1,m]}y_i^2}
    \end{gather*}
    \textbf{Lemma} for $z_1\leq z_2 \cdots \leq z_n = 0$, 
    \begin{gather*}
        \sum_{(i,j)\in E'_{-}}|z_i-z_j|\geq \phi \sum_{i=1}^{m}|z_i|
    \end{gather*}
    \textbf{Put it together}
    \begin{enumerate}
        \item Normalize: $\sum_{i=1}^m y_i^2 =1$
        \item Let $z_i=-y_i^2$, apply lemma to gather\begin{gather*}
            \sum_{(i,j)\in E_{-}'}|y_i^2-y_j^2|\geq \phi \sum_{i=1}^{m}|y_i|^2=\phi
        \end{gather*}
        \item Cauchy-Schwartz:\begin{gather*}
            \sum_{(i,j)\in E_{-}'}|y_i^2-y_j^2|\leq \Bigl( \sum_{(i,j)\in E_{-}'}(y_i-y_j)^2 \Bigr) ^{1/2}\Bigl(\sum_{(i,j)\in E_{-}'} (y_i+y_j)^2\Bigr)^{1/2}
        \end{gather*}
        \item \begin{gather*}
            \sum_{(i,j)\in E_{-}'}(y_i+y_j)^2 \leq 2\sum_{(i,j)\in E_{-}'}(y_i^2+y_j^2)\leq 2\sum_{i=1}^{m}d_{max}y_i^2\leq 2d_{max}
        \end{gather*}
        \item \begin{gather*}
            \frac{\sum_{(i,j)\in E_{-}'}(y_i-y_j)^2}{\sum_{i=0}^m y_i^2}\geq\frac{\bigl(\sum_{(i,j)\in E_{-}'}|y_i^2-y_j^2|\bigr)^2}{\sum_{(i,j)\in E_{-}'}(y_i+y_j)^2}\geq\frac{\phi^2}{2d_{max}}
        \end{gather*}
        \item \begin{gather*}
            \frac{x^TLx}{x^Tx}\geq\frac{y^TLy}{y^Ty}\geq\frac{\phi^2}{2d_{max}}
        \end{gather*}
    \end{enumerate}
    \subsection{Random Walks}
    \subsubsection{Definitions}
    Consider starting at some vertex $v \in V(G)$ and repeatedly moving to a neighboring vertex uniformly at random\\
    Let $p_t(u)$ denote the probability of being at a vertex at time t.
    \begin{gather*}
        \sum_{u \in V} p_t(u) = 1\\
        p_t(u) = \sum_{(u,v)\in E} p_{t-1}(v) \cdot \frac{1}{d(v)}
    \end{gather*}
    Define a matrix $W_G$ as follows:
    \begin{gather*}
        [W]_{i,j} = 
        \begin{cases}
            \frac{1}{d(j)} & (i,j) \in E\\
            0 & otherwise
        \end{cases}\\
        W = A \cdot D^{-1}
    \end{gather*}
    Where A is the adjacency matrix and D is diagonal matrix with degrees.\\
    \subsubsection{Stationary Distribution}
    Let $\pi$ be defined as a probability vector \textit{stationary distribution} of the random walk:
    \begin{gather*}
        \pi (u) = \frac{d(u)}{\sum_{v \in V} d(v)}
    \end{gather*}
    \textit{$\pi$ is a probability distribution}\\
    \begin{gather*}
        W \cdot \pi = \pi
    \end{gather*}
    Hence is at one point the random walk follows distribution $\pi$, it shall follow $\pi$ for all future steps. Or $\pi$ is 
    W's eigenvector with eigenvalue 1. Note that the probability need not converge to $\pi$ or at all.\\
    \subsubsection{Lazry Random Walks}
    \textit{Lazy Random Walks}
    \begin{itemize}
        \item With probability 1/2 stay at the current node
        \item With probability 1/2 take a step of the original walk
    \end{itemize}
    This breaks the periodicity of the walk in the following way
    \begin{gather*}
        W' = (W+I) /2=(I+A\cdot I)/2
    \end{gather*}
    However, since W and W' are not symmetric, it is difficult to analyze. \textit{Normalized Random Walk Matrix}
    \begin{align*}
        N &= D ^{-1/2} \cdot W \cdot D ^{1/2} &=&D ^{-1/2} \cdot A \cdot D^{-1/2}\\
        N' & = D ^{-1/2} \cdot W' \cdot D^{1/2}&=&(I+D^{-1/2}\cdot A \cdot D^{-1/2})/2
    \end{align*}
    N and W share eigenvalues and their eigenvectors are related.\\
    \indent If $v$ is eigenvector of N, then $q = D^{1/2}v$ is eigenvector of W\\
    \indent therefore because W has eigenvector $D \cdot \mathbf{1}$ with eigenvalue 1, N has eigenvector $D^{1/2}\cdot \mathbf{1}$
    with eigenvalue 1
    \subsubsection{Connection to Laplacians}
    The normalized laplacian $\mathcal{L}$ is defined as
    \begin{gather*}
        \mathcal{L}=D^{-1/2}\cdot L \cdot D^{-1/2}
    \end{gather*}
    and
    \begin{gather*}
        N=I-\mathcal{L}
    \end{gather*}
    Eigenvalues of N are given by $\mu=1-\lambda$ of the eigenvalues of $\mathcal{L}$, reorder them such that\\
    \begin{gather*}
        1=\mu_{1}\geq\mu_{2}\cdots\geq\mu_{n}
    \end{gather*}
    \textbf{Theorems}\\
    \begin{itemize}
        \item for all i, $\mu_{i}\in[-1, 1]$
        \item if G is connected, $\lambda_{2}<1$
        \item the eigenvalue of -1 only occurs for bipartite Graphs
    \end{itemize}
    For eigenvalues of N', $\mu_i'$
    \begin{itemize}
        \item for all i, $\mu_{i}'\in[0,1]$
        \item if G is connected, then $\mu_2'<1$
    \end{itemize}
    \subsubsection{$\ell_2$ Convergence}
    The spectral gap is defined as $\lambda=1-\mu_2$\\
    \textbf{$\ell_2$ Distance} for probability distribution p, q is defined to be $||p-q||_{2}=\sqrt{\sum_{i}(p(i)-q(i))^{2}}$
    , basically the Euclidean Distance.
    \textbf{Theorem} Let $p_0$ denote some initial probability distribution, and $p_t$ the distribution after t steps
    of lazy random walks. Then
    \begin{gather*}
        ||p_t-\pi||_{2}\leq(1-\lambda_2)^{2}\sqrt{\frac{max_{x} d(x)}{min_y d(y)}}
    \end{gather*}
    This is easily shown for regular graphs.
    \subsubsection{Conductance}
    Conductance controls how well-knit a graph is: how fast a random walk converges to the stationary distribution.
    Or how fast one can move from one part of a graph to another
    \textbf{Definition} For $S \subset V$, let
    \begin{gather*}
        \Phi(S) = \frac{e(S)}{min(\sum_{v \in S}d(v), \sum_{v\in \overline{S}}d(v)}
    \end{gather*}
    The conductance 
    \begin{gather*}
        \Phi(G) = \min_{S\subset V}\Phi(S)\\
        \Theta(1)\cdot\Phi^{2}(G) \leq 1-\mu_2'\leq\Theta(1)\cdot\Phi(G)
    \end{gather*}
    The above is analogous to Cheeger's Inequality.
    \subsubsection{Monte Carlo Methods}
    For ranomd variable $r\in{0,1}$ such that $\mathbf{Pr}[r=1]=p$ and $\mathbf{Pr}[r=0]=1-p$
    Assuming draw n independent $r_1,r_2\dots r_n$. Let $R =\sum_{i}r_i$. By linearity of expectations, $E[R]=n\cdot p$\\
    We  say R \textit{$\epsilon$-Approximates} $\mathbf{E}[R]$ if
    \begin{gather*}
        (1-\epsilon)\mathbf{E}[R]\leq R\leq (1+\epsilon)\mathbf{E}[R]
    \end{gather*}
    This error measure is multiplicative.\\
    \textbf{Chernoff Bound} The Probability that R fails to $\epsilon$ approximate \textbf{E}[R] is
    \begin{gather*}
        \mathbf{Pr}[|R-\mathbf{E}[R]| \geq \epsilon\mathbf{E}[R]] \leq 2e^{-np\epsilon ^{2}/12}=2e^{-\mathbf{E}[R]\epsilon^{2}/12}
    \end{gather*}
    \begin{itemize}
        \item The bound is nearly tight
        \item The trials need to be independent for the bound to hold
        \item The guarentee is multiplicative but not additive
        \item For fiexed $\epsilon$, it falls off exponentially in n. So one can improve from 1/2 to $1/2^k$ by doing
        k trials
        \item Therefore, smaller n requires more trials
        \item If a $\epsilon$-approximation is required with probability $1-\delta$,
        \begin{gather*}
            N\geq \Theta(\frac{log(1/\delta)}{p\epsilon^{2}})
        \end{gather*}
    \end{itemize}


    \newpage
    \chapter{Combinatorics}
    \section{Expanders and Eigenvalues}
    \subsection{Expander Graphs}
    The \textbf{edge expansion} or \textbf{Cheeger Constant} is
    \begin{gather}
        h(G) = \min_{|S|\leq n/2} \frac{e(S,\overline{S})}{|S|}
    \end{gather}
    A graph is \textbf{(d, $\epsilon$)-expander} if it is d-regular and $h(G)\geq \epsilon$
    \subsection{Random Graphs}
    A bipartite graph with $n+n$ vertices $L \cup R$ is a \textbf{(d, $\beta$)-expander} if the degrees in L are d and any set of vertex 
    $S \subset L$ of size $|S|\leq n/d$ has at least $\beta |S|$ neighbors in R.\\
    Let $d\geq 4$ and G be a bipartite graph obtained by randomly choosing d edges for each vertex in L, then G is a (d, d/4)-expander with constant positive probability.\\
    \begin{gather*}
        Pr[\exists S, T;|S|\leq n/d, T\leq \beta |S|] < \sum_{s=1}^{n/d}{n \choose s}{n \choose \beta s}(\frac{\beta s}{n})^{ds}
    \end{gather*}
    \subsection{Eigenvalue Bounds on Expansion}
    \begin{gather*}
        h(G) \geq \frac{1}{2}(d-\lambda_2)
    \end{gather*}
    \textbf{Proof} For any subset of the vertices S let $\mathbf{x}=(n-s)\mathbf{1}_S-s\mathbf{1}_{\overline{S}}$. Then $\mathbf{x^Tx}=(n-s)^2s+s^2(n-s)=s(n-s)n$.
    \begin{flalign*}
        \mathbf{x^T}A\mathbf{x} & =2\sum_{(i,j)\in E}x_ix_j=2(n-s)^2e(S)-2s(n-s)e(S,\overline{S})+2s^2e(\overline{S})\\
        & =(n-s)^2(ds-e(S,\overline{S}))-2s(n-s)e(S,\overline{S})+s^2(d(n-s)-e(S,\overline{S}))\\
        & =dns(n-s)-n^2e(S,\overline{S})
    \end{flalign*}
    \begin{gather*}
        \lambda_2\geq\frac{x^TAx}{x^Tx}=\frac{dns(n-s)-n^2e(S,\overline{S})}{s(n-s)n}=d-\frac{ne(S,\overline{S})}{s(n-s)}
    \end{gather*}
    For $s\leq \frac{n}{2}$,
    \begin{gather*}
        \frac{e(S,\overline{S})}{|S|}\geq\frac{n-s}{n}(d-\lambda_2)\geq\frac{1}{2}(d-\lambda_2)
    \end{gather*}
    $(d-\lambda_2)$ is called the \textbf{spectral gap}
    \begin{gather*}
        h(G)\leq\sqrt{d(d-\lambda_2)}
    \end{gather*}

    
\end{document}